\documentclass[10pt]{amsart}
\usepackage[a4paper]{geometry}
\usepackage{float}




\title{Small signal analysis for nonlinear evolution equations\\ Draft.}
\author{J. Fuhrmann}
\newcommand{\CV}{{\mathcal{V}}}
\newcommand{\VV}{\mathbb{V}}
\newcommand{\PP}{\mathbb{P}}
\newcommand{\TT}{\mathbb{T}}
\newcommand{\MM}{\mathbb{M}}
\newtheorem{example}{Example}

\begin{document}
\maketitle
%\begin{multicols}{2}
\section{Frequency response of an electrical network}
  
The  current  dependency  on   voltage  of  basic  basic  elements  of
electrical networks  can be subsummed  using the notion  of impedance.
Let $U$ be  a given voltage difference applied to  the device, and $I$
be the resulting current.  When  calculating the current response to a
periodic  voltage  perturbations, calculations  can  be simplified  by
switching  to  complex numbers.   So  we  assume  an applied  periodic
voltage $U(t)=U_a\exp(i\omega  t)$ with amplitude  $U_a$ and frequency
$\omega$, we get 

\begin{table}[H]
  \begin{center}
\renewcommand{\arraystretch}{1.4}
\begin{tabular}{lllll}
Circuit Element     & Law     & Standard form & Complex Form & Impedance\\ 
Resistance  & Ohm     & $ I(t)=\frac1R U(t)$                                & $I(t)=\frac1R U(t)$       &  $Z(\omega)=R$ \\
Capacity    & Faraday & $ I(t)=C(t)\frac{d}{dt}U(t)$                        & $I(t)=Ci\omega U(t)$&          $Z(\omega)=\frac1{Ci\omega}$\\
Inductivity & Henry   & $ I(t)=I_0+\frac1L\int\limits_{t_0}^tU(\tau) d\tau$ & $I(t)=\frac1{Li\omega}$ U(t)&$Z(\omega)=Li\omega$ \\
\end{tabular}
\caption{\label{tab:impedance} Impedance of three basic elements of electric circuits}\hfill
\end{center}
\end{table}

Any  network consisting  of  these  elements can  be  analyzed in  the
complex domain  using Kirchoffs law  and regarding the impedance  as a
complex resistance.

\section{Impedance spectroscopy in nonlinear evolution equations}

Usually, the interpretations of impedance spectroscopy measurements of
other  systems,  e.g. electrochemical  systems  is  performed using  a
replacement  circuit consisting of  electrical elements.   Being quite
successful in many cases,  this interpretation basically is limited to
compartment type models.

Here,  we discuss an  approach of  applying impedance  spectroscopy to
abstract evolution equations.

For a  given time  interval $\TT=[0,T]$, a  Banach space  $\CV$ called
{\em state space},  and a finite dimensional space  $\PP$, called {\em
parameter  space},  regard  the  abstract doubly  nonlinear  evolution
equation
\begin{equation}\label{eq:abstrevol}
 \frac{d S(v(t),\lambda)}{dt} + D(v(t),\lambda)=0
\end{equation}

The state of  the system is {\em measured} by  some functional $M: \CV
\rightarrow  \MM$,   where  $\MM$  is  the   finite  dimensional  {\em
measurement space}.

As an example, $\CV$ may  be a finite dimensional space containing the
solution of some discretized system of partial differential equations.


Given   a  steady   state,  $(v_0,   \lambda_0)$  such   that  $D(v_0,
\lambda_0)=0$, measured  by $M_0=M(v_0)$, we  would like to  trace its
response  to  a small,  periodic  perturbation $\lambda(t)=  \lambda_a
\exp(i\omega  t)   $.   Expressing   this  response  as   $M(t)=  M_0+
M_a(\omega)  \exp(i\omega  t)$,  we  yield the  impedance  $Z(\omega)=
M_a(\omega)^{-1} \lambda_a$.



In order to calculate the frequency response, we make the ansatz $v(t)=v_0+v_a\exp(i\omega  t)$ for
the perturbation of the state variable and calculate the first order Taylor expansions
the terms $S,D,R$:
\begin{equation*}
  \begin{split}
    S(v_0+v_a\exp(i\omega  t),\lambda_0+\lambda_a\exp(i\omega  t))\approx&S(v_0,\lambda_0)+ 
          S_v(v_0,\lambda_0)v_a\exp(i\omega  t)+
          S_\lambda(v_0,\lambda_0)\lambda_a\exp(i\omega  t)\\
    D(v_0+v_a\exp(i\omega  t),\lambda_0+\lambda_a\exp(i\omega  t))\approx&D(v_0,\lambda_0)+ 
          D_v(v_0,\lambda_0)v_a\exp(i\omega  t)+
          D_\lambda(v_0,\lambda_0)\lambda_a\exp(i\omega  t)\\
    M(v_0+v_a\exp(i\omega  t))\approx&M(v_0)+ 
          M_v(v_0)v_a\exp(i\omega  t)
  \end{split}
\end{equation*}

Putting them into equation \eqref{eq:abstrevol}  and using the steady state condition yields
\begin{equation*}
  \begin{split}
    \frac{d}{dt}\left( 
      S_v(v_0,\lambda_0)v_a\exp(i\omega  t)+
      S_\lambda(v_0,\lambda_0)\lambda_a\exp(i\omega  t)\right)+ 
    D_v(v_0,\lambda_0)v_a\exp(i\omega  t)+
    D_\lambda(v_0,\lambda_0)\lambda_a\exp(i\omega  t)&=0\\
    i\omega\left( 
      S_v(v_0,\lambda_0)v_a\exp(i\omega  t)+
      S_\lambda(v_0,\lambda_0)\lambda_a\exp(i\omega  t)\right)+ 
    D_v(v_0,\lambda_0)v_a\exp(i\omega  t)+
    D_\lambda(v_0,\lambda_0)\lambda_a\exp(i\omega  t)&=0\\
    i\omega\left( 
      S_v(v_0,\lambda_0)v_a+
      S_\lambda(v_0,\lambda_0)\lambda_a\right)+ 
    D_v(v_0,\lambda_0)v_a+
    D_\lambda(v_0,\lambda_0)\lambda_a&=0
  \end{split}
\end{equation*}

Assuming $\dim \MM =\dim \PP=1$ and dividing by $\lambda_a$, we arrive at solving 
\begin{equation*}
    i\omega\left( 
      S_v(v_0,\lambda_0)v_a+
      S_\lambda(v_0,\lambda_0)\right)+ 
    D_v(v_0,\lambda_0)v_a+
    D_\lambda(v_0,\lambda_0)=0
\end{equation*}
for given $\omega$ with the unknown $v_a$.
The impedance then can be calculated as 
\begin{equation*}
  Z(\omega)= \frac1{M_v(v_0)v_a}
\end{equation*}


The first example is a formal derivation of the use of Dirichlet boundary conditions.
\begin{example}{Dirichlet Boundary Conditions}\\
  Let $D(v,\lambda)= D_i(v)
  +\frac1\epsilon\delta_{\Gamma_0}(v)
  +\frac1\epsilon\delta_{\Gamma_1}(v-\lambda)$. This covers the case of an applie
  voltage.
Then  
\begin{equation*}
  \begin{split}
    D_v(v,\lambda)= D_{i,v}(v)+ \frac1\epsilon\delta_{\Gamma_0}+ \frac1\epsilon\delta_{\Gamma_1}\\
    D_\lambda(v,\lambda)= -\frac1\epsilon\delta_{\Gamma_1}\\
   \end{split}
\end{equation*}
Therefore we have to solve 
\begin{equation*}
  i\omega S_v v_a + D_{i,v} v_a ++\frac1\epsilon\delta_{\Gamma_0}v_a +\frac1\epsilon\delta_{\Gamma_1}(v_a-1) =0
\end{equation*}
which corresponds to the Dirichlet problem  for the linearized equation 
with boundary condition 1 on $\Gamma$.
In order to implement these boundary conditions, we use penalty methods, and therefore, we can carry
over this approach to the discrete case.
\end{example}

The next two examples can be taken for benchmarking numerical methods.
\begin{example}{Current Response}
  We calculate the impedance of the current response at $L$ to voltage change in $0$ of the linear reaction  diffusion system in $(0,L)$  
  \begin{equation*}
    \begin{cases}
      Cu_t - (Du_x)_x + Ru=&0\\
      u(0,t)=&\lambda\\
      u(L,t)=&0\\
    \end{cases}
  \end{equation*}
    As response functions we take the currents  $I_0=Du_x(0,t)$ and $I_L=Du_x(L,t)$.
    The corresponding impedance equation is
    \begin{equation*}
      \begin{cases}
        Ci\omega v - (Dv_x)_x +Rv =&0\\
        v(0,t)=&1\\
        v(L,t)=&0.\\
      \end{cases}
    \end{equation*}
Setting $z=\sqrt{i\omega\frac{C}{D}+\frac{R}{D}}$, for the solution, we make the ansatz
\begin{equation*}
  v=ae^{zx}+be^{-zx}
\end{equation*}
which fulfills the differential equation.
Setting $e^+=e^{zL},e^-=e^{-zL}$, from the boundary 
conditons we get the system
\begin{equation*}
  \begin{cases}
    a+b&=1\\
    ae^++be^-&=0\\
  \end{cases}
\end{equation*}
with the solutions $a=\frac{e^-}{e^--e^+},b=\frac{e^+}{e^+-e^-}$
Therefore, we have
\begin{equation*}
  Dv_x(0)=Dz(a-b)=Dz\frac{e^-+e^+}{e^--e^+}
\end{equation*}
and
\begin{equation*}
  Dv_x(L)=Dz(ae^+-be^-)=Dz\frac{e^-e^++e^+e^-}{e^--e^+}=\frac{2Dz}{e^--e^+}
\end{equation*}
\end{example}


\begin{example}{Voltage response}
Regard the system
  \begin{equation*}
    \begin{cases}
      Cu_t - (Du_x)_x + Ru=&0\\
      u(0,t)=&\lambda\\
      (Du_x)(L,t)=&0\\
    \end{cases}
  \end{equation*}
  and observe the values $u(L)$ and $Du_x(0)$.
The same ansatz as in the previous example leads to 
\begin{equation*}
  \begin{cases}
    a+b&=1\\
    izae^+-izbe^-&=0\\
  \end{cases}
\end{equation*}
with the solutions $a=\frac{e^-}{e^++e^-}$ and $b=\frac{e^+}{e^++e^-}$
Therefore,  $Du_x(0)=iDz\frac{e^--e^+}{e^++e^-}$ and
$u(L)=\frac2{e^++e^-}$.
\end{example}




Algorithmically, we proceed as follows:
\begin{quotation}
\parindent0pt
Given $\lambda$, solve steady state equation $D(v,\lambda)=0$\\
Obtain Jacobi matrix for stationary problem \\% A
Obtain Jacobi matrix of time derivative\\ % bprime

Prepare solution+rhs with boundary conditions\\
for $\omega=\omega_0\dots\omega_1$ do\\
.~~~~ Setup complex matrix\\
.~~~~ Solve complex system, take old solution as initial value\\
.~~~~ calculate functional\\
\end{quotation}

If the functional is linear, e.g. a boundary flux calculated by a test function, it is sufficient to
use the complex version of the already existing linear code.

\end{document}










